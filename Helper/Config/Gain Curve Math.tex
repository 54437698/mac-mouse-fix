(This is a .tex file so that vscode highlights the braces and brackets)

# Fourth version

See https://www.desmos.com/calculator/4qvpumdf4g

# This is a like 3rd version but we want to make the sens curve approach its cap smoothly instead of the derivative. So we want the constraint h'(v_1) = 0 instead of f''(v_1) = 0. So this is basically a correction for third version.
# Edit: This causes a weird arch in f'(x). Maybe version 3 is better after all. Edit2: I just don't like the curvature. Linear feels better. 4. or 3. doesn't really make a difference. 

# The relevant curves are (still)
# f(x) = (ax) + (bx)^2 + (cx)^3
# f''(x) = 2 (b^2 + 3 c^3 x)
# h(x) = ((ax) + (bx)^2 + (cx)^3) / x
# h'(x) = (a + 2 b^2 x + 3 c^3 x^2)/x - (a x + b^2 x^2 + c^3 x^3)/x^2

# Meta params:
#   s_0 -> min sens
#   s_1 -> max sens
#   v_1 -> The cap speed

# The following constraints should hold
#   h(0) = s_0
#   h(v_1) = s_1
#   h'(v_1) = 0

# Finding the curve params

# Solving h(0) = s_0 for a
# a = s_0

# Solving h(v_1) = s_1 for b
# b = sqrt(-a + c^3 (-v_1^2) + s_1)/sqrt(v_1)

# Solving h'(v_1) = 0 for c
# c = ((-1/2)^(1/3) b^(2/3))/(v_1)^(1/3)

# Plugging b into c and solving for c
# c = (a - s_1)^(1/3)/v_1^(2/3)
#   (We can interpolate between -c and c to control the curvature of h(x). Setting c=0 yields 0 curvature.)

# Third version

See https://www.desmos.com/calculator/hmaivjcfix (This is actually used for 4th version now I think?)

# In this version we want to define a point that the real sens curve passes through. Because f'(x) isn't actually the sens curve. It describes how much the pointer speed changes when you change the mouse speed for a given mouse speed. I guess you could call that `sens` but what we were trying to define is how high the outSpeed/inSpeed for a given inSpeed. f(x) is outSpeed(inSpeed) so the sens curve we were thinking about is just f(x)/x not f'(x). We call h(x) = f(x)/x

# The relevant curves are
# f(x) = (ax) + (bx)^2 + (cx)^3
# f''(x) = 2 (b^2 + 3 c^3 x)
# h(x) = ((ax) + (bx)^2 + (cx)^3) / x

# So now we define new meta params:
#   s_0 -> min sens
#   s_1 -> cap sens
#   v_1 -> The cap speed

# The following constraints should hold
#   h(0) = s_0
#   h(v_1) = s_1
#   f''(v_1) = 0


# Finding the curve params

# Solving h(0) = s_0 for a
# a = s_0

# Solving h(v_1) = s_1 for b
# b = sqrt(-a + c^3 (-v_1^2) + s_1)/sqrt(v_1)

# Solving f''(v_1) = 0 for c (alredy did that below)
# c = -b^(2/3)/(3^(1/3) (v_1)^(1/3))

# Plugging b into c and solving for c
#   1. c = -((-1/2)^(1/3) (a - s_1)^(1/3))/v_1^(2/3)
#   2. c = (a - s_1)^(1/3)/(2^(1/3) v_1^(2/3))
#   3. c = ((-1)^(2/3) (a - s_1)^(1/3))/(2^(1/3) v_1^(2/3))
#   -> They all are the same thing just with umformung. 2. Looks simplest though


# Second attempt

See https://www.desmos.com/calculator/usbskelvq5

# We try to find some c, such that the slope of the sens curve approaches the cap smoothly. (At the cap sens cap, the slope of the sens curve is zero.)

# Our original function is f(x) = (ax) + (bx)^2 + (cx)^3 + (dx)^4. 
# Input is mouse speed and output is cursor speed. f(x) = outSpeed(inSpeed)
# The derivative of f is the sens curve f'(x) = sens(inSpeed).
# The second derivative of f is the gain curve f''(x) = sensSlope(inSpeed)
# The Apple algorithm also lets us define a cap speed s_c. Such that for any x>s_c f'(x) = f'(s_c).
# At s_c the curve becomes a straight horizontal line.
# We want to cap the sens because it feels good and makes things more controllable. 

# So far we've been making the sens curve a straight line with a positive slope which then abruptly transitions into a straight line with slope 0 at cap speed s_c. 
#   Like this the gain curve was non-continuous

# Our goal now is to make the gain curve continuous. So to have the sens curve's slope change continuously instead of abruptly

# Find c such that the gain is continuous:

# To achieve this we need to use parameters a, b and c. (So far we've been setting c and d to 0) When c is negative the sens curve curves down. With the right c we can achieve a continuous gain curve.
# d doesn't help us

# So we start out with f(x) = (ax) + (bx)^2 + (cx)^3

# Now we get the derivates:
# f'(x) = a + x (2 b^2 + 3 c^3 x)
# f''(x) = 2 (b^2 + 3 c^3 x)

# We want to configure the curves with 3 meta parameters:
#  1. s_0 -> The min sens
#  2. v_1 -> The cap inputSpeed 
#  3. s_1 -> The max sens

# We want to choose a, b and c such that the following constraints hold:
#  1. f'(0) = s_0
#  2. f'(v_1) = s_1
#  3. f''(v_1) = 0

# Solving f'(0) = s_0 for a
#   a = s_0

# Solving f'(v_1) = s_1 for b:
#   b = ± sqrt(-a - 3 c^3 v_1^2 + s_1)/(sqrt(2) sqrt(v_1))
#   (Since b is squared anyways its sign doesn't matter)
#   b = sqrt(-a - 3 c^3 v_1^2 + s_1)/(sqrt(2) sqrt(v_1))

# Solving f''(v_1) = 0 for c:
#   (Roots of unity)
#   1. c = ((-1/3)^(1/3) b^(2/3))/(v_1)^(1/3) and v_1 !=0
#   2. c = -b^(2/3)/(3^(1/3) (v_1)^(1/3)) and v_1 !=0
#   3. c = -((-1)^(2/3) b^(2/3))/(3^(1/3) (v_1)^(1/3)) and v_1 !=0
#   The only real solution is 2.
#   c = -b^(2/3)/(3^(1/3) (v_1)^(1/3)) and v_1 !=0

# Note that our solution for b depends on c and our solution for c depends on b
# (b plugged into c)
# c = -(sqrt(-a - 3 c^3 v_1^2 + s_1)/(sqrt(2) sqrt(v_1)))^(2/3)/(3^(1/3) (v_1)^(1/3))
# (Solved for c again) (Roots of unity)
# 1. c = -((-1/3)^(1/3) (a - s_1)^(1/3))/v_1^(2/3)
# 2. c = (a - s_1)^(1/3)/(3^(1/3) v_1^(2/3))
# 3. c = ((-1)^(2/3) (a - s_1)^(1/3))/(3^(1/3) v_1^(2/3))
# The only real solution is 2. 
# c = (a - s_1)^(1/3)/(3^(1/3) v_1^(2/3))

# This works!! (See Desmos link above)

---------

# First attempt (Something went wrong. Doing again)


# Original function (plugged solution for b solution for c)

c=\frac{\sqrt[3]{-\frac{1}{6}}\sqrt[3]{0-2(\frac{\sqrt{-a-3c^3s_c^2+s_1}}{\sqrt{2}\sqrt{s_c}})^2}}{\sqrt[3]{s_c}}

# Line breaks and indentation

c=\frac{
    \sqrt[3]{
        -\frac{1}{6}
    }\sqrt[3]{
        0-2(\frac{\sqrt{
            -a-3c^3s_c^2+s_1
        }}{
            \sqrt{2}\sqrt{s_c}
        })^2
    }
}{
    \sqrt[3]{s_c}
}

# Remove latex fraction notation

c=(
    \sqrt[3]{
        -1/6
    }\sqrt[3]{
        -2((
            \sqrt{-a-3c^3s_c^2+s_1}
        )/(
            \sqrt{2}\sqrt{s_c}
        ))^2
    }
)/(
    \sqrt[3]{s_c}
)

# Remove latex root notation

c=(
    (-1/6)^(1/3)
    (
        -2((
            (-a-3c^3s_c^2+s_1)^0.5
        )/(
            2^0.5 s_c^0.5
        ))^2
    )^(1/3)
)/(
    s_c^(1/3)
)

# Rename s_c to s_3 to confues WA less

c=(
    (-1/6)^(1/3)
    (
        -2((
            (-a-3c^3s_3^2+s_1)^0.5
        )/(
            2^0.5 s_3^0.5
        ))^2
    )^(1/3)
)/(
    s_3^(1/3)
)

# WA finally solved it!!!
# It gives 3 solutions. WA phone app says these are the "3 roots of unity"

1. c = ((-1/3)^(2/3) (s_1 - a)^(1/3))/s_3^(2/3)

2. c = (s_1 - a)^(1/3)/(3^(2/3) s_3^(2/3))

3. c = -((-1)^(1/3) (s_1 - a)^(1/3))/(3^(2/3) s_3^(2/3))

# Replacing s_3 with s_c

1. c = ((-1/3)^(2/3) (s_1 - a)^(1/3))/s_c^(2/3)

2. c = (s_1 - a)^(1/3)/(3^(2/3) s_c^(2/3))

3. c = -((-1)^(1/3) (s_1 - a)^(1/3))/(3^(2/3) s_c^(2/3))x`'

# Trying solution 2. (real solution) -> doesn't quite work, the x axis Schnitt is negative instead of positive

# Trying solution 1. -> Exact same result as 2.

# Trying solution 3. -> Also exact same. 

# -> I must've entered some wrong equations
